\section{Installation Guide}
\label{sec:installation}

Follow these steps to set up the project environment on your local machine.

\subsection{Step 1: Unzip the Project}
Extract the provided project zip file to a directory of your choice (e.g., \texttt{/Users/username/Projects/shailaja-project}).

\subsection{Step 2: Create a Virtual Environment}
It is recommended to use a virtual environment to manage dependencies. Open your terminal, navigate to the project directory, and run:

\begin{verbatim}
# For macOS/Linux
python3 -m venv venv
source venv/bin/activate

# For Windows
python -m venv venv
.\venv\Scripts\activate
\end{verbatim}

\subsection{Step 3: Install Dependencies}
Once the virtual environment is active, install the required Python packages using \texttt{pip}:

\begin{verbatim}
pip install -r requirements.txt
\end{verbatim}

\subsection{Step 4: Verify Installation}
To verify that the installation was successful, you can check the installed packages:

\begin{verbatim}
pip list
\end{verbatim}

Ensure that packages like \texttt{pandas}, \texttt{scikit-learn}, and \texttt{streamlit} are listed.

\subsection{Step 5: Data Setup}
Before running any analysis, ensure that the raw data files are placed in the correct directories.
\begin{enumerate}
    \item Place the raw survey response CSV file in \texttt{data/raw/}.
    \item Place the experiment results CSV file in \texttt{data/synthetic/} (or \texttt{data/raw/} if using real experiment data).
\end{enumerate}

The preprocessing script is configured to look for specific filenames. If your filenames differ, you may need to update the paths in \texttt{src/processing/preprocess\_for\_modeling.py} or rename your files to match the expected defaults.

