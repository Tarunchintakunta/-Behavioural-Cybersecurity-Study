\section{Methodology}
\label{sec:methodology}

This chapter outlines the research methodology adopted to investigate the relationship between human factors and phishing vulnerability. It details the research design, data collection instruments, participant selection, and ethical considerations.

\subsection{Research Design}
The study employs a quantitative research design, utilizing a correlational and experimental approach. The correlational aspect involves assessing the relationship between independent variables (cognitive biases, stress, multitasking, demographics) and the dependent variable (phishing vulnerability). The experimental aspect involves a simulated phishing campaign to measure actual behavioral responses rather than relying solely on self-reported intentions.

This dual-method approach was chosen to overcome the limitations of purely survey-based studies, which often suffer from social desirability bias—where participants overstate their security awareness. By comparing survey responses with actual behavior in the simulation, the study aims to provide a more accurate assessment of vulnerability.

\begin{figure}[h]
    \centering
    \includegraphics[width=0.8\textwidth]{figures/mediation_model.png}
    \caption{Proposed Mediation Model}
    \label{fig:mediation}
\end{figure}

\subsection{Data Collection Instruments}
Data collection was conducted in two phases:

\subsubsection{Phase 1: Psychometric Survey}
A structured online questionnaire was designed to capture demographic data and psychological traits. The survey consisted of four sections:
\begin{enumerate}
    \item \textbf{Demographics}: Age, gender, education level, job role, and years of experience.
    \item \textbf{Cognitive Biases}: A series of scenario-based questions adapted from standard psychological scales to measure susceptibility to optimism bias, authority bias, and scarcity bias.
    \item \textbf{Situational Factors}: Self-assessment of current stress levels (using the Perceived Stress Scale) and multitasking habits (using the Media Multitasking Index).
    \item \textbf{Security Awareness}: Questions regarding prior cybersecurity training and general knowledge of phishing indicators.
\end{enumerate}

\subsubsection{Phase 2: Phishing Simulation}
Approximately two weeks after the survey, participants were subjected to a simulated phishing attack. The simulation involved sending a carefully crafted email mimicking a legitimate internal communication (e.g., "Urgent: Password Expiry Notification"). The email contained a link to a benign landing page that recorded whether the participant:
\begin{itemize}
    \item Opened the email.
    \item Clicked the link (indicating vulnerability).
    \item Reported the email to the IT security team (indicating resilience).
\end{itemize}

\subsection{Participants}
The study sample consisted of N=189 participants recruited from a mid-sized technology company. The selection criteria ensured a diverse representation of job roles, ranging from technical staff (developers, IT support) to non-technical staff (HR, finance, sales). Participation was voluntary, and no monetary incentives were provided to avoid skewing the results.

\subsection{Ethical Considerations}
Ethical approval was obtained from the university's ethics committee prior to data collection. Key ethical measures included:
\begin{itemize}
    \item \textbf{Informed Consent}: Participants were informed about the general nature of the study (a "cybersecurity behavior study") but were not explicitly told about the phishing simulation to maintain the element of surprise. A debriefing session was held after the experiment.
    \item \textbf{Anonymity}: All data was anonymized. Participant IDs were used to link survey responses with simulation results, ensuring that individual performance could not be traced back to specific employees by their management.
    \item \textbf{Data Protection}: All collected data was stored on encrypted servers and accessible only to the research team.
\end{itemize}
