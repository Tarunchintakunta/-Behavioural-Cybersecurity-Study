\begin{abstract}
Phishing attacks remain a predominant cybersecurity threat, exploiting human psychology rather than technical vulnerabilities. This study investigates the role of human factors—specifically cognitive biases, stress, multitasking, and digital literacy—in an individual's susceptibility to phishing. By combining a detailed behavioral survey with a simulated phishing experiment, we analyzed data from 189 participants to identify key predictors of vulnerability. Our findings reveal that high levels of stress and authority bias significantly increase the likelihood of clicking on phishing links, while prior cybersecurity training and high digital literacy offer protective effects. We developed a machine learning model to predict vulnerability based on these behavioral traits, achieving promising initial results. This research underscores the necessity of moving beyond "one-size-fits-all" security training towards personalized, behavior-aware interventions.
\end{abstract}