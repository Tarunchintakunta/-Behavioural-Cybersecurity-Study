\begin{abstract}
Phishing attacks remain a critical cybersecurity threat, exploiting human psychological vulnerabilities rather than technical flaws. This research investigates the impact of human factors—specifically cognitive biases, stress, multitasking, and demographics—on phishing susceptibility. Utilizing a quantitative approach, the study combined a psychometric survey with a simulated phishing experiment involving 189 participants to assess actual behavioral responses.

Statistical analysis revealed a strong positive correlation between high stress levels and increased vulnerability ($r=0.63$), confirming that cognitive load impairs security decision-making. Conversely, multitasking showed negligible impact. Notably, prior cybersecurity training did not significantly reduce susceptibility, suggesting traditional compliance-based training is ineffective. Demographic analysis indicated that specific job roles are stronger predictors of vulnerability than general traits.

Machine learning models were developed to predict high-risk individuals. While the Support Vector Machine (SVM) achieved 92\% accuracy, performance was constrained by class imbalance, highlighting the difficulty of modeling rare security events. These findings underscore the necessity of moving beyond ``one-size-fits-all'' training toward targeted, behavior-centric interventions that account for an employee's psychological state and organizational role.
\end{abstract}