\begin{abstract}
Phishing attacks remain one of the most prevalent and damaging cybersecurity threats, exploiting human psychological vulnerabilities rather than technical loopholes. Despite advancements in technical defenses, the "human factor" continues to be the weakest link in the security chain. This research investigates the role of human factors—specifically cognitive biases, stress, multitasking, and demographic characteristics—in susceptibility to phishing attacks. Adopting a quantitative approach, the study utilizes a dual-method data collection strategy involving a psychometric survey and a simulated phishing experiment with 189 participants.

The research aims to develop a predictive model for phishing vulnerability and analyze the correlation between psychological states and security behaviors. Key findings indicate a significant positive correlation between stress levels and phishing susceptibility, while multitasking showed a negligible effect. Surprisingly, prior cybersecurity training did not yield a statistically significant reduction in vulnerability, suggesting that current training paradigms may be insufficient. Demographic analysis revealed that certain job roles exhibit higher vulnerability, highlighting the need for tailored interventions.

Machine learning models, including Logistic Regression, Random Forest, Gradient Boosting, and Support Vector Machines (SVM), were trained to predict vulnerability. The SVM model achieved the highest cross-validation ROC AUC score of 0.38, though overall predictive performance on the test set was limited (Accuracy: 92\%, ROC AUC: 0.45), primarily due to class imbalance and the complexity of human behavior. This study contributes to the field of behavioral cybersecurity by providing empirical evidence of the psychological determinants of phishing susceptibility and offering recommendations for more effective, human-centric security awareness programs.
\end{abstract}