\section{Conclusion}
\label{sec:conclusion}

\subsection{Summary of Contributions}
This research set out to investigate the role of human factors in phishing susceptibility, moving beyond the traditional technical-centric view of cybersecurity. By designing a study that combined psychometric assessment with a realistic phishing simulation, this project has provided empirical evidence on how psychological states and demographic factors influence security behavior.

The key contributions of this study are:
\begin{enumerate}
    \item \textbf{Validation of the Stress-Vulnerability Link}: The study confirmed a strong positive correlation between perceived stress and phishing susceptibility, highlighting the need for organizations to consider employee well-being as a security factor.
    \item \textbf{Critique of Current Training Models}: The finding that prior training had no significant impact on vulnerability challenges the effectiveness of standard compliance-based training and calls for more engaging, behavioral-focused interventions.
    \item \textbf{Role-Based Risk Profiling}: The identification of significant differences in vulnerability across job roles provides a data-driven basis for targeted security policies.
    \item \textbf{Benchmark for ML Prediction}: While the predictive models achieved limited success due to data constraints, the study established a baseline and identified key challenges (class imbalance, feature selection) for future behavioral modeling.
\end{enumerate}

\subsection{Limitations}
Despite the rigorous methodology, several limitations must be acknowledged:
\begin{itemize}
    \item \textbf{Sample Size}: With 189 participants, the study had limited statistical power to detect smaller effects, particularly for complex interactions between variables.
    \item \textbf{Synthetic Nature of Experiment}: Although designed to be realistic, the phishing simulation was still a controlled exercise. In a real-world attack, the consequences (and thus the psychological pressure) would be higher.
    \item \textbf{Self-Report Bias}: The reliance on self-reported measures for stress and multitasking may have introduced bias, as participants might underreport negative traits.
\end{itemize}

\subsection{Future Work}
Building on the findings of this research, several avenues for future work are proposed:
\begin{itemize}
    \item \textbf{Longitudinal Studies}: Conducting a study over a longer period to track how vulnerability changes with fluctuating stress levels and repeated training interventions.
    \item \textbf{Advanced Machine Learning}: Exploring deep learning techniques or anomaly detection algorithms (e.g., Isolation Forest) to better handle the class imbalance and capture non-linear relationships.
    \item \textbf{Real-Time Intervention}: Developing a browser plugin that uses real-time behavioral indicators (e.g., mouse movement, typing speed) to detect stress or distraction and provide "just-in-time" warnings when a user interacts with a suspicious email.
\end{itemize}

\subsection{Final Remarks}
Cybersecurity is a shared responsibility that extends beyond the IT department. As technical defenses become more impenetrable, attackers will continue to target the human mind. This research underscores that the solution lies not just in better firewalls, but in a better understanding of ourselves—our biases, our limits, and our behaviors. By integrating psychological insights into security strategy, we can build a more resilient human firewall.
