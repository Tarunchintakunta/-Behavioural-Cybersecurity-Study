\section{Conclusion and Future Work}
\label{sec:conclusion}

\subsection{Conclusion}
This study successfully demonstrated that psychological and behavioral factors—specifically cognitive biases, stress, and digital literacy—are significant predictors of phishing vulnerability. By integrating these factors into a machine learning model, we achieved a predictive accuracy of 91\%, significantly outperforming random baselines. The findings suggest that the "human firewall" can be strengthened not just by technical training, but by addressing the underlying behavioral vulnerabilities.

The strong correlation between Authority Bias and click-through rates highlights a critical flaw in current organizational cultures where questioning authority is discouraged. Furthermore, the impact of stress and multitasking confirms that security is often a victim of the "attention economy."

\subsection{Limitations}
The study has a few limitations that must be acknowledged:
\begin{itemize}
    \item \textbf{Sample Size}: The initial real-world dataset (N=21) was small, necessitating the use of synthetic data augmentation. While statistically valid, future studies should aim for larger primary datasets.
    \item \textbf{Simulation Realism}: The phishing simulation, while realistic, was conducted in a controlled environment. Participants might behave differently in a high-stakes real-world scenario.
    \item \textbf{Self-Reporting Bias}: Data on stress and biases relied on self-reporting, which can be subject to social desirability bias.
\end{itemize}

\subsection{Future Work}
Future research should focus on:
\begin{itemize}
    \item \textbf{Longitudinal Studies}: Tracking participants over a 6-12 month period to observe how vulnerability fluctuates with changing stress levels and project deadlines.
    \item \textbf{Real-time Intervention Tools}: Developing a browser plugin that uses the predictive model to warn users in real-time when they are exhibiting high-risk behaviors (e.g., "You seem stressed and are reading an email from an unknown sender. Please slow down.").
    \item \textbf{Cross-Cultural Analysis}: Expanding the demographic scope to include participants from different cultural backgrounds to validate the universality of these behavioral factors.
\end{itemize}
