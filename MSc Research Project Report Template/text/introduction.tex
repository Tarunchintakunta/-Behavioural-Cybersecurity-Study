\section{Introduction}
\label{sec:intro}

\subsection{Background}
In the digital age, information security has become a paramount concern for individuals and organizations alike. While technological defenses such as firewalls, intrusion detection systems, and encryption have matured significantly, cybercriminals have increasingly shifted their focus towards the "human element" of security. Phishing, a form of social engineering where attackers deceive victims into revealing sensitive information or installing malware, stands as a testament to this shift. According to the 2024 Verizon Data Breach Investigations Report, the human element is involved in 68\% of all breaches, with phishing being a primary vector \citep{Verizon2024}.

The success of phishing attacks relies heavily on exploiting human psychology. Attackers leverage cognitive biases, urgency, and authority to bypass critical thinking and elicit immediate, often detrimental, responses. For instance, an email mimicking a CEO's urgent request for a wire transfer triggers a compliance response that overrides standard verification procedures. This phenomenon underscores that cybersecurity is not merely a technical challenge but a behavioral one. Understanding why individuals fall for phishing—beyond simple lack of knowledge—is crucial for developing robust defenses.

\subsection{Problem Statement}
Despite the widespread implementation of security awareness training and technical filters, phishing success rates remain alarmingly high. Traditional approaches often treat all users as equally vulnerable or focus solely on knowledge gaps. However, research suggests that susceptibility is dynamic and influenced by transient psychological states (e.g., stress, fatigue) and inherent cognitive traits (e.g., impulsivity, optimism bias).

Current predictive models often fail to account for these nuanced human factors, relying instead on static demographic data or past behavior logs. There is a critical gap in understanding how specific psychological stressors and cognitive biases interact to increase vulnerability in real-time. Furthermore, the effectiveness of generic "one-size-fits-all" training programs is increasingly questioned, as they fail to address the underlying behavioral drivers of unsafe actions. Without a deeper understanding of these human factors, organizations remain exposed to sophisticated social engineering attacks that bypass technical perimeters.

\subsection{Research Questions}
To address the identified problem, this research poses the following key questions:

\begin{itemize}
    \item \textbf{RQ1}: Is there a significant correlation between specific cognitive biases (e.g., optimism bias, authority bias) and an individual's susceptibility to phishing attacks?
    \item \textbf{RQ2}: To what extent do situational factors, specifically high stress levels and multitasking habits, increase the likelihood of falling for a phishing attempt?
    \item \textbf{RQ3}: Does prior cybersecurity training significantly reduce phishing vulnerability, and is this effect consistent across different demographic groups?
    \item \textbf{RQ4}: Can a machine learning model accurately predict an individual's phishing vulnerability based on a combination of demographic, psychological, and behavioral features?
\end{itemize}

\subsection{Research Objectives}
The primary objective of this study is to empirically investigate the determinants of phishing susceptibility and develop a predictive framework for identifying high-risk individuals. The specific objectives are:

\begin{enumerate}
    \item To conduct a comprehensive literature review on behavioral cybersecurity, focusing on the intersection of psychology and information security.
    \item To design and execute a data collection methodology that combines psychometric surveys with a simulated phishing experiment to capture both self-reported traits and actual behavioral responses.
    \item To analyze the collected data using statistical methods (Correlation, T-tests, ANOVA) to validate the relationships between human factors and phishing vulnerability.
    \item To develop and evaluate machine learning models (Logistic Regression, Random Forest, SVM, GBM) to predict susceptibility and identify the most predictive features.
    \item To provide actionable recommendations for organizations to enhance their security posture through targeted interventions and human-centric policies.
\end{enumerate}

\subsection{Scope and Limitations}
This research focuses on email-based phishing attacks within a corporate or organizational context. It specifically examines the influence of cognitive biases, stress, multitasking, and basic demographics. Technical aspects of phishing detection (e.g., email header analysis, URL reputation) are outside the scope of this study.

The study is subject to certain limitations. Firstly, the sample size (N=189) may limit the generalizability of the findings to the broader population. Secondly, the use of a simulated environment, while necessary for ethical reasons, may not perfectly replicate the psychological pressure of a real-world attack. Finally, the reliance on self-reported data for psychological states introduces the possibility of response bias.

\subsection{Report Structure}
The remainder of this report is organized as follows:
\begin{itemize}
    \item \textbf{Chapter 2: Related Work} reviews the existing literature on social engineering, human factors in security, and previous attempts at modeling phishing susceptibility.
    \item \textbf{Chapter 3: Methodology} details the research design, data collection instruments, and ethical considerations.
    \item \textbf{Chapter 4: Design and Architecture} describes the system architecture, data processing pipeline, and feature engineering steps.
    \item \textbf{Chapter 5: Implementation} outlines the technical implementation of the machine learning models and the analysis framework.
    \item \textbf{Chapter 6: Evaluation} presents the statistical and machine learning results, discussing their implications in relation to the research questions.
    \item \textbf{Chapter 7: Conclusion} summarizes the key contributions, discusses limitations, and suggests directions for future research.
\end{itemize}
