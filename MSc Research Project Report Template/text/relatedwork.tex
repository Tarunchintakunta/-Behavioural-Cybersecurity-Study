\section{Related Work}
\label{sec:related_work}

This chapter reviews the state-of-the-art in behavioral cybersecurity, examining the evolution of phishing attacks, the psychological mechanisms of susceptibility, and the role of machine learning in detection. It integrates recent findings from 2020-2024 to provide a contemporary context for the study.

\subsection{The Evolving Threat Landscape}
Phishing has evolved from generic "spray-and-pray" campaigns to highly sophisticated, targeted operations. \cite{Alkhalil2023} describe a new anatomy of phishing attacks that leverages Artificial Intelligence (AI) to automate social engineering. Attackers now use Generative AI to craft grammatically perfect and contextually relevant emails, bypassing traditional "bad grammar" heuristics that users were trained to spot.

The \cite{Verizon2024} Data Breach Investigations Report highlights that the human element remains the primary driver of breaches, involved in 68\% of all incidents. Similarly, \cite{Proofpoint2024} report that 71\% of organizations were infected with ransomware due to a successful phishing attack, underscoring the critical need for human-centric defenses.

\subsection{Psychological Determinants of Susceptibility}
Recent research has shifted focus from demographic profiling to understanding the underlying psychological traits that make individuals vulnerable.

\subsubsection{Cognitive Biases and Heuristics}
\cite{Trellix2024} emphasize that attackers exploit "System 1" thinking—fast, intuitive, and emotional processing. Their "Mind of the CISO" report notes that fatigue and burnout significantly degrade an employee's ability to switch to "System 2" analytical thinking when evaluating suspicious emails. This aligns with recent systematic reviews by \cite{Rocha2021} on the role of cognitive biases in security decision-making.

\subsubsection{Trust and Impulsivity}

\cite{Alkhalil2021} identify trust and impulsivity as key predictors of susceptibility. Users who score high on agreeableness are more likely to comply with requests from perceived authority figures. Furthermore, \cite{Oest2022} found in a longitudinal study that susceptibility is not static; users fluctuate between vulnerable and resilient states depending on their immediate environment and stress levels.

\subsubsection{Personality Traits}
Beyond transient states, stable personality traits also play a role. \cite{Workman2020} and \cite{Iqbal2021} explored the "Big Five" personality traits, finding that Extraversion and Openness are often positively correlated with phishing susceptibility due to higher levels of online social interaction and curiosity. Conversely, \cite{Lawson2022} highlight Conscientiousness as a protective factor, associated with more cautious digital behavior.

\subsection{Situational Factors: The Role of Stress}
The impact of situational stressors has gained attention in the post-pandemic era of hybrid work. \cite{DArcy2022} demonstrated a direct link between workplace stress and reduced compliance with security protocols. High cognitive load, exacerbated by multitasking and constant notifications, depletes the mental resources required for vigilance. This study builds on this by quantifying the correlation between perceived stress and the likelihood of clicking a phishing link.

\subsection{Machine Learning in Phishing Detection}
While human education is vital, technical defenses remain the first line of defense. \cite{Basit2023} provide a comprehensive survey of AI-enabled detection techniques, noting that while Deep Learning models (e.g., CNNs, RNNs) achieve high accuracy in detecting malicious URLs, they often lack explainability. \cite{Sahingoz2020} and \cite{Tang2022} demonstrate the effectiveness of Random Forest and SVM in classifying phishing URLs based on lexical features, achieving accuracies exceeding 95\%.

\cite{Yousef2023} proposed a hybrid machine learning model that combines content-based features (e.g., keyword analysis) with context-based features (e.g., sender reputation). However, most existing models focus on the \textit{email} itself rather than the \textit{recipient}. There is a gap in the literature for "User Vulnerability Prediction" models that use behavioral data to predict \textit{who} is most likely to fall for an attack, allowing for proactive, targeted intervention. \cite{Zamani2024} emphasize the need for such user-centric approaches to complement traditional filtering.

\subsection{Gap Analysis}
Despite the wealth of research, a disconnect remains between technical detection systems and behavioral insights. Most studies either focus solely on the psychology (without building predictive models) or solely on the algorithms (ignoring the human user). This research aims to bridge this gap by developing a predictive framework that incorporates psychological, situational, and demographic features to identify high-risk individuals.